\documentclass[a4paper,11pt]{article}

\usepackage{physics}
\usepackage{natbib}
\usepackage[top=2.2cm, bottom=1.8cm, left=2.8cm, right=2.3cm]{geometry}
\usepackage[T1]{fontenc} %for å bruke æøå
\usepackage[utf8]{inputenc}
\usepackage{graphicx} %for å inkludere grafikk
\usepackage{caption}
\usepackage{verbatim} %for å inkludere filer med tegn LaTeX ikke liker
\usepackage{mathpazo}
\usepackage{amsmath,amsthm,amssymb}
\usepackage{hyperref}% Booksmarks hyperref
\usepackage{bookmark}% Booksmarks 
\usepackage{float} % placing figugre in HERE \begin{table}[H] OR \begin{figure}[H]
\usepackage{booktabs} % table
\usepackage{verbatim} % to have comment in paragraph\\
\usepackage{multirow}
\bibliographystyle{plain}

%\begin{figure}[hbtp]
%\includegraphics[scale=0.4]{test1.pdf}
%\caption{Exact and numerial solutions for $n=10$ mesh points.} 
%\label{fig:n10points}
%\end{figure}

\renewcommand{\thesubsubsection}{\Roman{subsubsection}}

\begin{document}
	
\title{FYS2160 obligatory assignment 2}
\author{Anders Eriksen}
\maketitle


%%%%%%%%%%%  SECTION 1 %%%%%%%%%%%%%
\section{Van der Waals gas}
Helmholtz free energy for van der Waals gas given as
\[ F( T, V, N ) = - N k T \qty( \ln(\frac{  V - Nb  }{ N\Lambda^3 } ) + 1 ) - \frac{ a N^2 }{ V } \]
\[ \Lambda( T ) = \sqrt{ \frac{ h^2 }{ 2\pi m k T } }  \]
$\Lambda$ is the thermal wavelength as a pure functio of temperature. 

\subsection{ Derive equation of state for gas, expression for pressure $P( V, T, N )$ as func of 
            $V, T, N$ from the Helmholtz free energy above. }%%%%%%%%%%
To find the equation of state for the van der Waal gas, we need to use the thermodynamic identity of the Helmholtz free
energy, 
\[ dF = - SdT - PdV + \mu dN \]
And we know that the Helmholtz free energy is in a system that is isolated with regards to volume and molecular movement, but
not for heat. With the assumption of a quasi static change, the temperature will never leave that of the thermal bath. Thus, $dT=0$.
We also assume that the amount of gas we are looking at is unchanging. Thus, $dN=0$ as well. What remains of our identity, is
\begin{align*}
    dF|_{T,N} = - PdV \quad &\rightarrow \quad P = -\dv{F}{V}|_{T,N} \\
    \dv{V}\qty( -NkT\ln( V - Nb) - aN^2\frac{1}{V} ) &\rightarrow -\qty( -\frac{NkT}{V - Nb} + aN^2\frac{1}{V^2} ) = P \\
\end{align*}
\[ P( V, T, N ) =  \frac{NkT}{V - Nb} - aN^2\frac{1}{V^2} \]\\


%%%%%%%%%%
One mole van der Waals gas undergoes reversible isothermal compression from volume $V_1$ to $V_2 < V_1$
%%%%%%%%%

\subsection{ I) use $P\qty(V, T N)$ to find expression for work done on gas. II) does the lower density 
            limit require more or less work than the ideal gass? III) how so for the high density limit IV) why}%%%%%%%%%%

\subsubsection{}
For a gass, the work done is
\[ W = - \int_{V_2}^{V_1} dV P \]
where the negative is a result of work being done on the gas to compress it. 
\[ W = \int_{v_2}^{V_1} dV \dv{F}{V}|_{T,N} \]
 which is F from initial to finalvalues, excluding T and N dependencies and solves to
\[ - \qty( NkT\ln( V - Nb) - aN^2\frac{1}{V^2} )_{V_2}^{V_1} \]
\[ \qty( NkT\ln(\frac{V_2 - Nb}{V_1 - Nb}) - aN^2\qty(\frac{1}{V_2^2} - \frac{1}{V_1^2} ) ) \]


\subsection{ Apply van der Waals gas to nitrogen ($N_2$). Critical parameters $P_c = 33.6atm$, 
            $V_c = 0.089\frac{l}{mol}$, $T_c = 126 K$. Plot $P - V$ isotherms for $T = 77, 100, 110, 115, 120, 125 K$. }%%%%%%%%%%

\begin{figure}[hbtp]
\includegraphics[scale=0.75]{E13Isotherm.png}
\caption{Here we see the isotherms with regards to a van der Waals gas, with dimensionless variables rho, v and t equal to 
        $\frac{P}{P_c}$, $\frac{V}{V_c}$, $\frac{T}{T_c}$ respectively}
 \label{isotherm}
\end{figure}

\subsection{ With construction of Maxwell equal area (as presented during lecture), determine liquid volume $V_l(T)$
            and gas volume $V_g(T)$ for given temperature. Estimate by eye and find approximate position of temperature
            where areas are equal. } % See lecture notes 13 for the area!%%%%%%%%%%

Using the Maxwell equal area, we approximate the areas between a line and the graph, I did this using a line drawn after 
approximating the equal areas for the relevant curve. 

\begin{tabular}{ccc}
\hline
\multicolumn{1}{c}{ Temperature / K } & \multicolumn{1}{c}{ $V_l$ / atm } & \multicolumn{1}{c}{ $V_g$ / atm } \\
\hline
100 &  0.107 &  0.400      \\ % 1.199,  4.497
110 &  0.096 &  0.262      \\ % 1.077, 2.949
115 &  0.003 &  0.205      \\ % 0.037, 2.303
120 &  0.092 &  0.153      \\ % 1.030, 1.715
125 &  0.089 &  0.108      \\ % 0.998, 1.217
\hline
    \label{VlVg}
\
\end{tabular}\\

After having read off the varous liquid and gass volumes, we rescaled them with through multiplying with the critical volume given above. The result is 
the table \ref{VlVg}

\subsection{ Recall areas for a critical temperature. find $T_c$, i.e plot $V_l(T) - V_g(T)$ as a function of T. 
            compare with theoretical temperature. Why do they differ? }%%%%%%%%%%

We Made a small sampled difference between the liquid and gas volumes, before doing a linear fit with scipy's stat.linregress. This gave us 
$~ V_l - V_g \approx 0.011\cdot T - 1.377$ which means that the critical temperature hits at around $T = 125 K$, which is only $1 K$ off of the theoretical
value. This is a rather good fit, considering we only made eye measurements and a fit of 5 points.

%%%%%%%%%%%%%%%%%%%%%%%%%%%%%%%%%%%

%%%%%%%%%%%% SECTION 2 %%%%%%%%%%%%%%%
\section{ Partition function }
Particle with 5 states, each with energy $\epsilon_i = -0.1, -0.05, 0, 0,05, 0.1 eV$ for $i\in \qty[1, 5]$. Thermal 
equilibrium at room tepmerature T = 300K

\subsection{ Partition function for particle $Z_1$ }%%%%%%%%%%

We know from Schroeder, that the partition function $Z = \sum_i e^{-\beta \epsilon_i}$, where $\beta = \frac{1}{kT}$. 
We have energies in electron volt eV, and can express the Boltzmann constant as $k = 8.617\cdot 10^{-5} \frac{eV}{K}$.
This gives us a $\beta = \frac{1}{kT} = \frac{1}{6.617\cdot 3}\cdot 10^{5 - 2} = 38.7 \frac{1}{eV}$
For 1 particle, we can express the partition function as 
\[ Z = \sum_i e^{-\beta \epsilon_i} = e^{0.1\beta} + e^{0.05\beta} + 1 + e^{-0.05\beta} + e^{-0.1\beta} = 5.000 \].


\subsection{ $\Pr(\epsilon_i)$ for each i. }%%%%%%%%%%

Looking once more to Schroeder, we can find the probability of a state 
\[ P(\epsilon_i ) = \frac{1}{Z} e^{\beta \epsilon_i} \]
For the states, this gives \\
\begin{tabular}{ccc}
\hline
    \multicolumn{1}{c}{ $\epsilon_i$ / eV } & \multicolumn{1}{c}{ $P(\epsilon_i )$ } \\
\hline
  -0.10     &     0.201    \\ 
  -0.05     &     0.200    \\ 
   0.00     &     0.200    \\ 
   0.05     &     0.200    \\ 
   0.10     &     0.199    \\ 
\hline
    \label{probs}
\
\end{tabular}\\


\subsection{ The reference point for energies is arbitrary. Reshuffle states so we start at 0. repeat 1. and 2. for
            $\epsilon_i = 0, 0.05, 0.1, 0.15, 0.2$. What will remain and what will change? }%%%%%%%%%%
We retain our $\beta =  38.7 \frac{1}{eV}$, but the partition function is now

\[ Z = \sum_i e^{-\beta \epsilon_i} = 1 + e^{-0.05\beta} + e^{-0.1\beta} + e^{-0.15\beta} + e^{-0.2\beta} = 4.980 \].

The function for probability remains the same. 
\[ P(\epsilon_i ) = \frac{1}{Z} e^{\beta \epsilon_i } \]

\begin{tabular}{ccc}
\hline
    \multicolumn{1}{c}{ $\epsilon_i$ / eV } & \multicolumn{1}{c}{ $P(\epsilon_i )$ } \\
\hline
  -0.10     &      0.200   \\ 
  -0.05     &      0.200   \\ 
   0.00     &      0.199   \\ 
   0.05     &      0.199   \\ 
   0.10     &      0.199   \\ 
\hline
    \label{probs}
\
\end{tabular}\\

While we can see that there are some small changes here, I believe these arise from numericali inacuracies in my calculations. 
When checeking the total probability, I got a difference of $0.003\dots$, with $P1 = 0.999\dots$ and $P2 = 0.996\dots$. If I reduce
to 2 leading digits, there should be no difference, and both should sum to 1.

%%%%%%%%%%%%%%%%%%%%%%%%%%%%

%%%%%%%%%% SECTION 3 %%%%%%%%%%
\section{ Maxwell-Boltzmann distribution }
Maxwell-Boltzmann distribution of speed $D(v)$ for nitrogen gas $N_2$ in 3 dimensions.

\subsection{ find average speed  $\expval{v}$, root mean squared speed $v_{rms}$ and most probable speed $v_p(T)$ for $N_2$
            molecule at temperatures $T = 300$ and $600 K$ (molar mass of $N_2$ is $M_m = 0.028 \frac{kg}{mol}$) }%%%%%%%%%%

We can find on page 245, that if all velocities are allowed, then the integral of the average velocity can be expressed as
\[ \ev{v(T)} = \sqrt{ \frac{ 8kT }{ \pi m } } \]
Which takes the temperature and molecule mass into account. A heavy molecule could have the same energy with lower speed, and
a higher general energy would lead to a larger average speed, with an expected greater spread of velocities. With the given molar 
mass, 1 molecule should weig $\frac{1}{N_A}$ of that. Thus, 

\[ m = \frac{ M_{N_2, mol} }{ N_A } = \frac{ 2.8 }{ 6 } \cdot 10^{-25} = 4.67\cdot 10^{-26} kg \]
adding in a Boltzmann constant of $k_B = 1.38064\cdot 10^{-23}$ and a temperature $T = 3\cdot 10^{2}$. We get
\[ \ev{v(T)} = \sqrt{ \frac{ 8 * 1.38065 * 3 }{ \pi * 4.67 }\cdot 10^{-23+2-(-26)} } 
    = \sqrt{ 22.6 \cdot 10^6 } = 4.74 \cdot 10^3 \frac{m}{s} \]
Or roughly \underline{$\ev{v(T)} = 4.74\frac{km}{s}$}. And for temperature $T = 6\cdot 10^2$, we get 
\[ \ev{v(T)} = \sqrt{ \frac{ 8 * 1.38065 * 6 }{ \pi * 4.67 }\cdot 10^{-23+2-(-26)} } 
    = \sqrt{ 45.17\cdot 10^6 } = 6.72 \cdot 10^3 \frac{m}{s} \]
Or \underline{$\ev{v(T)} = 6.72 \frac{km}{s}$}

And the most probable speed $v_p(T)$, or $v_{max}(T)$ as Schroeder calls it, p. 244\cite{ThermalSchroeder}, given by
\[ v_p(T) = \sqrt{ \frac{ 2kT }{ m } } = \sqrt{ \frac{ 2\cdot 1.38064\cdot 3 }{ 4.67 } \cdot 10^{-23 + 2 + 26} } 
    = \sqrt{ 17.74 \cdot 10^{6} } = 4.21\cdot 10^3 \frac{m}{s}\]
Or \underline{$4.21\frac{km}{s}$}, slightly below the average, as expected. For $T = 6\cdot 10^2$, we have the same performance
as previously, scaling the result by a factor $\sqrt{2}$, leaving us with \underline{$v_p(T) = 5.95 \frac{km}{s}$}.


\subsection{ Plot Maxwell-Boltzmann distribution $D(v)$ for the 2 temperatures. in the same graph }%%%%%%%%%%

To plot the 2 graphs, we made a function to find the distribution values of a set of velocities given a temperature, setting everything else 
into different constants. After feeding the velocity range and the temperatures into the funciton, we made a normalized copy of the lowest temperature
distribution for later use. 

\begin{figure}[hbtp]
\includegraphics[scale=0.75]{E32MBdist.png}
\caption{Maxwell Boltzmann disrtributions of velocities in the range $0$ to $2\frac{km}{s}$, where the amplitude was scaled as shown,
    in order to avoid numerical inacuracies. We can see that for a similar amount of particles, an increase in energy will shift and widen
    the graph, with the overall velocity increasing as per the kinetic energy, but also that more particles can be found with velocities above
    and below the most likely.}
 \label{Maxwell-Boltzmann}
\end{figure}


\subsection{ Room temperature $T = 300K$, what fraction of $N_2$ molecules at speeds less than $300\frac{m}{s}$ }%%%%%%%%%%

To find the fraction of particles below $300\frac{m}{s}$, we need to take the integral of the distribution function from no velocity, to
$300$, 
\[ \int_0^{300}dv\quad D(v) =  \int_0^{300}dv\quad \beta v^2e^{-\alpha v^2} 
    \quad \leftarrow \quad \alpha = \frac{ m }{ 2kT } \; , \; \beta = \qty( \frac{ \alpha }{ \pi } )^{3/2} \] 
This becomes quite hairy, however. As an alternative I normalized the distribution with regards to $300K$ temperature and summed up the 
contributions up to velocity 300. This resulted in $20.2\%$ of particles having a velocity below $300 \frac{m}{s}$. Looking at 
figure \ref{Maxwell-Boltzmann}, we can see that the peak of the lowest temperature lies at about $4-500 \frac{m}{s}$. and there is a quick dropoff. 
The general amount within a standard deviation is $60 - 70\%$, and there is a far slower taper on the higher end of velocities. Thus, of the 
$30-40\%$ outside of this, it is not untoward to expect half or less to be on the lower velocities. The standard deviation seems to be at about 
$250\frac{m}{s}$ which is a little below our cutoff. This should pull the probability upwards some, but the majority of velocities will still be above
this. I therefore believe the estimation to be accurate. 


\subsection{ Temperature of Earth's upper atmosphere around $T=1kK$. calculate probability of $N_2$ molecule with velocity 
            greater than the earth's escape velocity of $11\frac{km}{s}$. Comment. (hint: calculate integral numerically) }%%%%%%%%%%

to calculate the probability of particles reaching escape velocity in the earth's atmosdphere, we made use of the functionality of the code for
previous exercises, generalizing it to handle the current problem. first generating the distribution at the given temperature and ensuring that the 
velocities covered go a fair bit beyond the target. Running over the normalized distribution, we check that the velociy is above our target and if so 
add up the contributions. This reaches up to a ratio of $10^{-88}$. The chance of a particle reaching escape velocity is astronomically small. It doesn't 
happen very often. 



\subsection{ repeat 4. for $H_2$ and $He$. Discuss results. ($M_{H_2, mol} = 0.002\frac{kg}{mol}$ and 
            $M_{He, mol} = 0.004\frac{kg}{mol}$)  }%%%%%%%%%%
Using once more the funcitonality of the program, we modify the distribution "constant" called "B" in the program to the relative mass of hydrogen gas and
Helium atoms with regards to Nitrogen. This allows us to run the same tests and fairly easily extract rates of $10^{6}$ and $10^{-12}$ respectively. These
two are significantly more likely, but it is still not a high likelihood. About $1$ in every million hydrogen atoms could escape, while only another millionth
of helium atoms escape. This can be seen in the differing masses, and how much energy can be stored in the motion of these molecules. 

\subsection{ consider the moon. calculate the probability of $N_2$ molecules to reach the moons escape velocity of
            $2.4\frac{km}{s}$. Assume $T=1kK$ as above. Use to explain the moon's lack of atmosphere. }%%%%%%%%%%
Changing the escape velocity to that of the moon, yet keeping the $N_2$ molecule scaling on the distribution gives a probability of $~10^{-4}$
to reach the stars. this is, in other words several orders of magnitude easier than even the lightest gasses on earth and roughly $84$ orders of 
magnitude easier than with $N_2$ on Earth. This means that an atmosphere on the moon would need to be very cool indeed if it was to retain any sort
of atmosphere. In addition, if you were to harbour dreams of leaping to the stars and have the choise between the Earth and Moon, go for the moon. 

%%%%%%%%%%%%%%%%%%%%%%%%%%%%%%


\bibliography{\string~/Documents/bibliography/Bibliography}
\end{document}
