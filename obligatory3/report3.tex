\documentclass[a4paper,11pt]{article}

\usepackage{physics}
\usepackage{natbib}
\usepackage[top=2.2cm, bottom=1.8cm, left=2.8cm, right=2.3cm]{geometry}
\usepackage[T1]{fontenc} %for å bruke æøå
\usepackage[utf8]{inputenc}
\usepackage{graphicx} %for å inkludere grafikk
\usepackage{caption}
\usepackage{verbatim} %for å inkludere filer med tegn LaTeX ikke liker
\usepackage{mathpazo}
\usepackage{amsmath,amsthm,amssymb}
\usepackage{hyperref}% Booksmarks hyperref
\usepackage{bookmark}% Booksmarks 
\usepackage{float} % placing figugre in HERE \begin{table}[H] OR \begin{figure}[H]
\usepackage{booktabs} % table
\usepackage{verbatim} % to have comment in paragraph\\
\usepackage{multirow}
\bibliographystyle{plain}

%\begin{figure}[hbtp]
%\includegraphics[scale=0.4]{test1.pdf}
%\caption{Exact and numerial solutions for $n=10$ mesh points.} 
%\label{fig:n10points}
%\end{figure}

\renewcommand{\thesubsubsection}{\Roman{subsubsection}}

\begin{document}
	
\title{FYS2160 obligatory assignment 2}
\author{Anders Eriksen}
\maketitle


%%%%%%%%%%%  SECTION 1 %%%%%%%%%%%%%
\section{Van der Waals gas}
Helmholtz free energy for van der Waals gas given as
\[ F( T, V, N ) = - N k T \qty( \ln(\frac{  V - Nb  }{ N\Lambda^3 } ) + 1 ) - \frac{ a N^2 }{ V } \]
\[ \Lambda( T ) = \sqrt{ \frac{ h^2 }{ 2\pi m k T } }  \]
$\Lambda$ is the thermal wavelength as a pure functio of temperature. 

\subsection{ Derive equation of state for gas, expression for pressure $P( V, T, N )$ as func of 
            $V, T, N$ from the Helmholtz free energy above. }%%%%%%%%%%
To find the equation of state for the van der Waal gas, we need to use the thermodynamic identity of the Helmholtz free
energy, 
\[ dF = - SdT - PdV + \mu dN \]
And we know that the Helmholtz free energy is in a system that is isolated with regards to volume and molecular movement, but
not for heat. With the assumption of a quasi static change, the temperature will never leave that of the thermal bath. Thus, $dT=0$.
We also assume that the amount of gas we are looking at is unchanging. Thus, $dN=0$ as well. What remains of our identity, is
\begin{align*}
    dF|_{T,N} = - PdV \quad &\rightarrow \quad P = -\dv{F}{V}|_{T,N} \\
    \dv{V}\qty( -NkT\ln( V - Nb) - aN^2\frac{1}{V} ) &\rightarrow -\qty( -\frac{NkT}{V - Nb} + aN^2\frac{1}{V} ) = P \\
\end{align*}
\[ P( V, T, N ) =  \frac{NkT}{V - Nb} - aN^2\frac{1}{V} \]\\


%%%%%%%%%%
One mole van der Waals gas undergoes reversible isothermal compression from volume $V_1$ to $V_2 < V_1$
%%%%%%%%%

\subsection{ I) use $P\qty(V, T N)$ to find expression for work done on gas. II) does the lower density 
            limit require more or less work than the ideal gass? III) how so for the high density limit IV) why}%%%%%%%%%%

\subsubsection{}
For a gass, the work done is
\[ W = -\int_{V_2}^{V_1} dV P \]
where the negative is a result of work being done on the gas to compress it. 
\[ W = \int_{v_2}^{V_1} dV \dv{F}{V}|_{T,N} \]
 which is F from initial to finalvalues, excluding T and N dependencies and solves to
\[ - \qty( NkT\ln( V - Nb) - aN^2\frac{1}{V} )_{V_2}^{V_1} \]
\[ \qty( NkT\ln(\frac{V_2 - Nb}{V_1 - Nb}) - aN^2\qty(\frac{1}{V_2} - \frac{1}{V_1} ) ) \]


\subsection{ Apply van der Waals gas to nitrogen ($N_2$). Critical parameters $P_c = 33.6atm$, 
            $V_c = 0.089\frac{l}{mol}$, $T_c = 126 K$. Plot $P - V$ isotherms for $T = 77, 100, 110, 115, 120, 125 K$. }%%%%%%%%%%


\subsection{ With construction of Maxwell equal area (as presented during lecture), determine liquid volume $V_l(T)$
            and gas volume $V_g(T)$ for given temperature. Estimate by eye and find approximate position of temperature
            where areas are equal. } % See lecture notes 13 for the area!%%%%%%%%%%


\subsection{ Recall areas for a critical temperature. find $T_c$, i.e plot $V_l(T) - V_g(T)$ as a function of T. 
            compare with theoretical temperature. Why do they differ? }%%%%%%%%%%

%%%%%%%%%%%%%%%%%%%%%%%%%%%%%%%%%%%

%%%%%%%%%%%% SECTION 2 %%%%%%%%%%%%%%%
\section{ Partition function }
Particle with 5 states, each with energy $\epsilon_i = -0.1, -0.05, 0, 0,05, 0.1 eV$ for $i\in \qty[1, 5]$. Thermal 
equilibrium at room tepmerature T = 300K

\subsection{ Partition function for particle $Z_1$ }%%%%%%%%%%


\subsection{ $\Pr(\epsilon_i)$ for each i. }%%%%%%%%%%


\subsection{ The reference point for energies is arbitrary. Reshuffle states so we start at 0. repeat 1. and 2. for
            $\epsilon_i = 0, 0.05, 0.1, 0.15, 0.2$. What will remain and what will change? }%%%%%%%%%%

%%%%%%%%%%%%%%%%%%%%%%%%%%%%

%%%%%%%%%% SECTION 3 %%%%%%%%%%
\section{ Maxwell-Boltzmann distribution }
Maxwell-Boltzmann distribution of speed $D(v)$ for nitrogen gas $N_2$ in 3 dimensions.

\subsection{ find average speed  $\expval{v}$, root mean squared speed $v_{rms}$ and most probable speed $v_p(T)$ for $N_2$
            molecule at temperatures $T = 300$ and $600 K$ (molar mass of $N_2$ is $M_m = 0.028 \frac{kg}{mol}$) }%%%%%%%%%%

We can find on page 245, that if all velocities are allowed, then the integral of the average velocity can be expressed as
\[ \ev{v(T)} = \sqrt{ \frac{ 8kT }{ \pi m } } \]
Which takes the temperature and molecule mass into account. A heavy molecule could have the same energy with lower speed, and
a higher general energy would lead to a larger average speed, with an expected greater spread of velocities. With the given molar 
mass, 1 molecule should weig $\frac{1}{N_A}$ of that. Thus, 

\[ m = \frac{ M_{N_2, mol} }{ N_A } = \frac{ 2.8 }{ 6 } \cdot 10^{-25} = 4.67\cdot 10^{-26} kg \]
adding in a Boltzmann constant of $k_B = 1.38064\cdot 10^{-23}$ and a temperature $T = 3\cdot 10^{2}$. We get
\[ \ev{v(T)} = \sqrt{ \frac{ 8 * 1.38065 * 3 }{ \pi * 4.67 }\cdot 10^{-23+2-(-26)} } 
    = \sqrt{ 22.6 \cdot 10^6 } = 4.74 \cdot 10^3 \frac{m}{s} \]
Or roughly \underline{$\ev{v(T)} = 4.74\frac{km}{s}$}. And for temperature $T = 6\cdot 10^2$, we get 
\[ \ev{v(T)} = \sqrt{ \frac{ 8 * 1.38065 * 6 }{ \pi * 4.67 }\cdot 10^{-23+2-(-26)} } 
    = \sqrt{ 45.17\cdot 10^6 } = 6.72 \cdot 10^3 \frac{m}{s} \]
Or \underline{$\ev{v(T)} = 6.72 \frac{km}{s}$}

And the most probable speed $v_p(T)$, or $v_{max}(T)$ as Schroeder calls it, p. 244\cite{ThermalSchroeder}, given by
\[ v_p(T) = \sqrt{ \frac{ 2kT }{ m } } = \sqrt{ \frac{ 2\cdot 1.38064\cdot 3 }{ 4.67 } \cdot 10^{-23 + 2 + 26} } 
    = \sqrt{ 17.74 \cdot 10^{6} } = 4.21\cdot 10^3 \frac{m}{s}\]
Or \underline{$4.21\frac{km}{s}$}, slightly below the average, as expected. For $T = 6\cdot 10^2$, we have the same performance
as previously, scaling the result by a factor $\sqrt{2}$, leaving us with \underline{$v_p(T) = 5.95 \frac{km}{s}$}.


\subsection{ Plot Maxwell-Boltzmann distribution $D(v)$ for the 2 temperatures. in the same graph }%%%%%%%%%%


\begin{figure}[hbtp]
\includegraphics[scale=0.75]{E32MBdist.png}
\caption{Maxwell Boltzmann disrtributions of velocities in the range $0$ to $2\frac{km}{s}$, where the amplitude was scaled as shown,
    in order to avoid numerical inacuracies. We can see that for a similar amount of particles, an increase in energy will shift and widen
    the graph, with the overall velocity increasing as per the kinetic energy, but also that more particles can be found with velocities above
    and below the most likely.} 
\label{fig:n10points}
\end{figure}


\subsection{ Room temperature $T = 300K$, what fraction of $N_2$ molecules at speeds less than $300\frac{m}{s}$ }%%%%%%%%%%

To find the fraction of particles below $300\frac{m}{s}$, we need to take the integral of the distribution function from no velocity, to
$300$, 
%\[ \int_0^{300}dv\quad D(v) = \alpha \int_0^{300}dv\quad v^2e^{-\beta v^2} 
%    \quad \leftarrow \quad \alpha = 4\pi \qty(\frac{m}{2\pi kT})^{3/2} \; \beta = \frac{m}{2kT} \]
\[ \int_0^{300}dv\quad D(v) =  \int_0^{300}dv\quad \beta v^2e^{-\alpha v^2} 
    \quad \leftarrow \quad \alpha = \frac{ m }{ 2kT } \; , \; \beta = \qty( \frac{ \alpha }{ \pi } )^{3/2} \] 


\subsection{ Temperature of Earth's upper atmosphere aroun $T=1kK$. calculate probability of $N_2$ molecule with velocity 
            greater than the earth's escape velocity of $11\frac{km}{s}$. Comment. (hint: calculate integral numerically) }%%%%%%%%%%


\subsection{ repeat 4. for $H_2$ and $He$. Discuss results. ($M_{H_2, mol} = 0.002\frac{kg}{mol}$ and 
            $M_{He, mol} = 0.004\frac{kg}{mol}$)  }%%%%%%%%%%


\subsection{ consider the moon. calculate the probability of $N_2$ molecules to reach the moons escape velocity of
            $2.4\frac{km}{s}$. Assume $T=1kK$ as above. Use to explain the moon's lack of atmosphere. }%%%%%%%%%%

%%%%%%%%%%%%%%%%%%%%%%%%%%%%%%


\bibliography{\string~/Documents/bibliography/Bibliography}
\end{document}
